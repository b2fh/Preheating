\documentclass[a4paper,11pt]{article}
\pdfoutput=1
\usepackage{jcappub}

\usepackage{amsmath,amssymb}
\usepackage{graphics,graphicx}

\graphicspath{{figures/},{lattice_scripts/}}

\newcommand{\figref}[1]{Fig.~\ref{#1}}

\begin{document}

\title{Preheating with Noncanonical Couplings after Large-Field Inflation}
\author[a]{J. Richard Bond}
\author[a,b]{Jonathan Braden}
\author[c]{Andrei Frolov}
\author[a]{Zhiqi Huang}
\affiliation[a]{CITA}
\affiliation[b]{Dept. of Physics, U of T}
\affiliation[c]{Simon Fraser University}

\emailAdd{bond@cita.utoronto.ca}
\emailAdd{jbraden@cita.utoronto.ca}
\emailAdd{frolov@sfu.ca}
\emailAdd{zqhuang@cita.utoronto.ca}

\abstract{BLAH BLAH}
\date{\today}
\maketitle

\section{Introduction}
BLAH BLAH BLAH
\begin{itemize}
  \item Given a large b-mode detection, assume that given a collection of conformal transformations, field redefinitions, etc. we can obtain an effective single-field description of the inflaton which is of the same simple monomial form near the minimum of the potential as on the inflationary plateau.
  \item Given this assumption, we will consider various couplings between the inflaton and some additional scalar degrees of freedom
\end{itemize}
Without a specific high-energy model for the inflaton, one cannot in general connect the post-inflationary potential to the inflationary plateau that is relevant when physically observable scales were exiting the horizon.
However, a detection of $r$ indicates that the field traversed a Planckian distance in field space.
It is not unreasonable to assume that after inflation the field will undergo damped oscillations around the minimum of its potential with an initial amplitude that is Planckian.
A natural question is then what the implications of such large amplitude oscillations are for any fields that couple to the inflaton.
This question has been extensively explored for potential couplings between the inflaton and a few other fields that we will denote as preheat fields.
However, much less is known about the dynamiccs in the case that there are many preheat fields, or many additional fields coupled to the preheat fields but not directly to the inflaton.
As well, the effect of couplings through noncanonical kinetic terms are also relatively unexplored.

BLAH BLAH BLAH
First we'll consider noncanonical couplings between the inflaton and a single preheat field.
Second, we'll look at (potential) couplings between the inflaton and many fields and the inflaton and a single field which itself couples to many other fields.

\section{Noncanonical Couplings to Preheat Fields}
In this first section we consider some specific examples from the class of Lagrangians
\begin{equation}
  \frac{\mathcal{L}}{\sqrt{g}} = -\frac{\partial_\mu\phi\partial^\mu\phi}{2} - V^{inf}(\phi) - G(\phi)\frac{\partial_\mu\chi\partial^\mu\chi}{2} - W(\chi) - U_{coup}(\phi,\chi)
\end{equation}
where we assume $V_\phi(\phi) \sim \phi^q$ and $\phi_{end} \sim M_P$.
Our assumption of large amplitude oscillations for the inflaton $\phi$ means that initially we will have $\phi \approx \phi_0f(t)/a^p$ with $\phi_0 \sim M_P$, $f(t)$ a periodic function and $p$ the exponent describing damping of the fluctuations due to expansion.

These types of actions can arise after an appropriate conformal transformation and field redefinition for nonminimal gravitational couplings, or between a modulus field $\phi$ and axion $\chi$.
{\bf Do these models give large field inflation with detectable $r$?}
The case $G(\phi)=1$ and $U_{coup}(\phi,\chi)$ some low order polynomial in the fields has been well studied in the literature and leads to a range of interesting instabilities for the field fluctuations.
%As some specific examples that lead to these types of actions, we could have $\phi$ nonminimally coupled to gravity $\Omega^2(\phi)\frac{M_P^2}{2}R$, in which case $G=\Omega^{-2}$ and $W+U = \omega^{-4}(W_{bare}+U_{bare})$.
%Some widespread examples that lead to such couplings are nonminimal coupling between the inflaton $\phi$ and the Ricci scalar, supergravity descriptions of inflation, and couplings between moduli and axion fields in string theory.  {\bf Are the last two really independent?}

%Given these theoretical motivations, we will consider three general classes of couplings between the inflaton and $\chi$
%\begin{itemize}
%  \item Coupling only via the noncanonical kinetic term
%  \item Coupling via an overall multiplier of the $\chi$ Lagrangian (as happens with nonminimally coupled gravity models in the Einstein frame (thus enforcing a relationship between the kinetic and potential couplings)
%  \item Direct coupling through a potential not associated with an overall multiplier
%\end{itemize}
%Most nonlinear studies of preheating have focussed on only the third option (but see~\ref{Zhiqi_modular,Giblin_gauge}.
%Within these broad classes, we can further distinguish cases where $\chi$ has it's own potential

\section{Noncanonical Couplings to an Otherwise Free Field}
Since potential couplings have been the object of many past studies, 
we will first focus on models with $U_{coup} = 0$ and $W(\chi) = m_\chi^2\chi^2/2$.
To maintain some small level of generality, we will consider several possible forms for the noncanonical coupling G:
$G(\phi) = (1+\xi\phi^2)^{-1}, G(\phi)=(1+\xi\phi^2), G(\phi)=e^{\gamma\phi}$ and $G(\phi)=e^{\gamma\phi^2/2}$.
The equation for fluctuations in $\chi$ becomes
\begin{equation}
  \delta\ddot{\chi} + \left(3H + (\partial_\phi\ln G)\dot{\phi}_{bg}\right)\delta\dot{\chi} + \left(\frac{k^2}{a^2} + G^{-1}(\phi)\partial_{\chi\chi}W \right)\delta\chi = 0
\end{equation}
with $\dot{} = d/dt$ a derivative with respect to cosmic time.
When $\phi$ evolves in a $\lambda\phi^4$ potential, it is instead convenient to work in conformal time where the $\phi$ oscillations are periodic
\begin{equation}
  \delta\chi'' + \left(2\mathcal{H} + (\partial_\phi\ln G)\phi_{bg}'\right)\delta\chi' + \left(k^2 + a^2G^{-1}(\phi)\partial_{\chi\chi}W \right)\delta\chi = 0
\end{equation}
where $'$ denotes a derivative with respect to conformal time and $\mathcal{H} = a'/a$.
Ignoring for the moment the expansion of the universe, fluctuations in $\chi$ behave as a damped harmonic oscillator with periodically varying coefficients.
Therefore, we can apply Floquet theory to search for unstable solutions $\delta\chi \sim e^{\mu t}f(t)$ where $f(t)$ is a periodic function.
This analysis can be performed on the equations as written, where it is clear that the oscillating damping term can cause exponential growth.
However, to make contact with the rest of the literature, we will define a normalized fluctuation variable $X_k \equiv a^p\sqrt{G(\phi)}\delta\chi_k$ where $p=3/2 (1)$ in cosmic (conformal) time.
$X_k$ then obeys the equation for a harmonic oscillator with a time-dependent frequency $X_k'' + \omega_k^2(t)X_k=0$ with
\begin{equation}
  \omega_k^2 = \frac{k^2}{a^2} - \ddot{\phi}_{bg}\partial_\phi\ln\sqrt{G} - \left( (\partial_\phi\ln\sqrt{G})^2 + \partial_{\phi\phi}\ln\sqrt{G}\right)\dot{\phi}_{bg}^2 + G^{-1}(\phi)\partial_{\chi\chi}W - \frac{3}{4}\left( H^2+2\frac{\ddot{a}}{a} \right)
\end{equation}
in cosmic time and
\begin{equation}
  \omega_k^2 = k^2 - \left(\phi_{bg}'' + \mathcal{H}\phi_{bg}'\right)\partial_\phi\ln\sqrt{G} - \left( (\partial_\phi\ln\sqrt{G})^2 + \partial_{\phi\phi}\ln\sqrt{G}\right)\phi_{bg}'^2 - \frac{a''}{a}
\end{equation}
in conformal time.
It is convenient to define the effective mass induced by the noncanonical coupling by
\begin{equation}
  m^2_{eff,G} \equiv -\ddot{\phi}_{bg}\partial_\phi\ln(\sqrt{G}) - \left( (\partial_\phi\ln\sqrt{G})^2 + \partial_{\phi\phi}\ln\sqrt{G}\right)\dot{\phi}_{bg}^2
  \label{eqn:eff_mass_G}
\end{equation}
As expected, the oscillating damping term has lead to a contribution to an oscillating effective frequency for the linear fluctuations.
In~\figref{fig:eff_frequencies_sine_osc} we plot this contribution to the effective frequency, assuming the background oscillation is of the form $\phi_{bg}=\phi_0\cos(t)$, for each of our four noncanonical couplings $G(\phi)$.
There are periods during which $m_{eff,G}^2 < 0$, indicating that tachyonic resonance can drive the production of fluctuations at $k^2 \ll a^2m_\phi^2$ in addition to broad band parametric resonance.
When the $\phi$ instead oscillates in a $\lambda\phi^4/4$ potential, 
the oscillations are instead of the form $a\phi(\tau) \approx \phi_0\mathrm{cn}\left(\sqrt{\lambda}M_P\tau|2^{-1}\right)$ with $\mathrm{cn}$ the Jacobi elliptic cosine function.
The dominant spectral contribution to $\mathrm{cn}$ comes from the first harmonic, so that the resulting $m_{eff,G}^2$ is very similar to those in~\figref{fig:eff_frequencies_sine_osc}.
\begin{figure}[h]
  \includegraphics[width=0.24\linewidth]{{{effmass_sinebg_exponential}}}
  \includegraphics[width=0.24\linewidth]{{{effmass_sinebg_gaussian}}}
  \includegraphics[width=0.24\linewidth]{{{effmass_sinebg_running}}}
  \includegraphics[width=0.24\linewidth]{{{effmass_sinebg_flatten}}}
  \caption{Effective mass squared induced by the noncanonical coupling ($m^2_{eff,G}$ defined in~\eqref{eqn:eff_mass_G}) for $\phi_{bg} = \phi_0\sin(t)$ and several choices of noncanonical coupling to $\chi$.  In all cases we have ignored expansion of the universe, so that $\phi_0$ is a constant.}
  \label{fig:eff_frequencies_sine_osc}
\end{figure}

In order to assess the transfer of energy from the inflaton $\phi$ into the preheat field $\chi$, it is useful to define
\begin{align}
  \rho_\chi &= G(\phi)\left(\frac{\dot{\chi}^2}{2} + \frac{(\nabla\chi)^2}{2a^2}\right) + W(\chi) \notag \\
  \rho_\phi &= \frac{\dot{\phi}^2}{2} + \frac{(\nabla\phi)^2}{2a^2} + V^{inf}(\phi)
  \label{eqn:rho_chi_phi}
\end{align}
Similarly, in order to measure the transfer of energy from the homogeneous condensate into fluctuations, it is also useful to consider the energy density stored in field gradients
\begin{equation}
  \rho_{grad} = \frac{1}{2a^2}\left((\nabla\phi)^2 + G(\phi)(\nabla\chi)^2\right)
\end{equation}

\subsection{$G(\phi) = e^{2\gamma\phi}$}
\begin{equation}
  G(\phi) = e^{2\gamma\phi}
\end{equation}
This gives
\begin{equation}
  \omega_k^2 = k^2 - \gamma\ddot{\phi}_{bg} - \gamma^2\dot{\phi}_{bg}^2
  \label{eqn:frequency_exponential}
\end{equation}
\begin{figure}
%  \includegraphics[width=0.3\linewidth]{{{floquet_sine_exponential_mass0}}}
  \includegraphics[width=0.32\linewidth]{{{floquet_sine_exponential_mass0.01}}}
  \includegraphics[width=0.32\linewidth]{{{floquet_sine_exponential_mass0.1}}}
  \includegraphics[width=0.32\linewidth]{{{floquet_sine_exponential_mass1}}}

  \caption{Floquet chart for~\eqref{eqn:frequency_exponential} assuming background oscillations $\phi_{bg} = \phi_0\cos(mt)$ for several choices of mass for the $\chi$ field.}
  \label{fig:floquet_exponential}
\end{figure}

\begin{figure}
%  \includegraphics[width=0.45\linewidth]{{{energy_grad_vary_gamma_n256}}}
  \includegraphics[width=0.45\linewidth]{{{energy_grad_vary_gamma_exponential_l4_n128_l20}}}
  \includegraphics[width=0.45\linewidth]{{{energy_grad_vary_gamma_exponential_m2_n128_l20}}}
  \caption{Transfer of energy into the preheat field $\chi$ and fluctuations as we vary $\gamma M_P$.  The solid lines are the fraction of energy stored in the $\chi$ field with $\rho_\chi = e^{\gamma\phi}\left(\frac{\dot{\chi}^2}{2}+\frac{(\nabla\chi)^2}{2a^2}\right) + m^2_\chi\chi^2/2$.  The dashed semitransparent lines are the fraction of energy stored in field gradients.  The left figure shows results for $V(\phi)=\lambda\phi^4/4$ and the the right panel for $V(\phi)=m^2\phi^2/2$.  All simulations in the left panel used a lattice with $N_{lat}=128^3$ lattice sites and a comoving side length $\sqrt{\lambda}M_PL=20$.  The runs in the right panel instead used $N_{lat}=128^3$ and $m_\phi L = 10$.}
\end{figure}

\begin{figure}
  \includegraphics[width=0.32\linewidth]{{{weff_exponent_alpha3_l20_phi4}}}
  \includegraphics[width=0.32\linewidth]{{{phi_spec_exponent_alpha3_l20_phi4}}}
  \includegraphics[width=0.32\linewidth]{{{chi_spec_exponent_alpha3_l20_phi4}}} \\
  \includegraphics[width=0.32\linewidth]{{{weff_exponent_alpha5_l20_phi4}}}
  \includegraphics[width=0.32\linewidth]{{{phi_spec_exponent_alpha5_l20_phi4}}}
  \includegraphics[width=0.32\linewidth]{{{chi_spec_exponent_alpha5_l20_phi4}}} \\
  \includegraphics[width=0.32\linewidth]{{{weff_exponent_alpha7_l20_phi4}}}
  \includegraphics[width=0.32\linewidth]{{{phi_spec_exponent_alpha7_l20_phi4}}}
  \includegraphics[width=0.32\linewidth]{{{chi_spec_exponent_alpha7_l20_phi4}}}
  \caption{Evolution of the equation of state $w=\bar{P}/\bar{\rho}$ and the field fluctuations for $\gamma M_P= $ 1.5 (\emph{top row}), 2.5 (\emph{middle row}), 3.5 (\emph{top row}) and $V(\phi) = \lambda\phi^4/4$.  $P_\alpha$ stands for the (unnormalized) power spectrum of $\alpha$.  $\rho_\chi$ is defined in~\eqref{eqn:rho_chi_phi}.  For the fluctuation spectra, for the first four oscillations of $\phi$ we have plotted the spectra at time steps $\sqrt{\lambda}M_P d\tau=0.5$, and $\sqrt{\lambda} M_P d\tau = $ for the remainder of the evolution.  As well, we have color-coded the lines based on the oscillations of $\phi$ with blue, red, green and purple corresponding to the first, second, third and fourth oscillations respectively.  Earlier times are in a lighter shade than later times.  The evolution after the fourth oscillation are colored in shades of orange.  In the top two rows, the amplification of $\chi$ terminates in the linear regime, and the self-resonance of $\phi$ is the dominant mechanism for fluctuation generation that eventually leads to the breakup of the inflaton condensate.  Meanwhile, for $\gamma M_P = $3.5, the tachyonic instability in $\chi$ causes the fluctuations to become nonlinear after a few oscillations of $phi$.}
  \label{fig:exponent_spectra_phi4}
\end{figure}

\begin{figure}
  \includegraphics[width=0.32\linewidth]{{{weff_exponent_alpha10_l20}}}
  \includegraphics[width=0.32\linewidth]{{{phi_spec_exponent_alpha10_l20}}}
  \includegraphics[width=0.32\linewidth]{{{chi_spec_exponent_alpha10_l20}}} \\

  \includegraphics[width=0.32\linewidth]{{{weff_exponent_alpha12.5_l10}}}
  \includegraphics[width=0.32\linewidth]{{{phi_spec_exponent_alpha12.5_l10}}}
  \includegraphics[width=0.32\linewidth]{{{chi_spec_exponent_alpha12.5_l10}}}

  \caption{Evolution of the equation of state and field fluctuation spectra for the exponential coupling $G(\phi)=e^{2\gamma\phi}$ and $V(\phi) = m^2\phi^2 / 2$.  As with that background oscillations for the $\phi^4$ potential, we have $\rho_\phi = \frac{\dot{\phi}^2}{2a^2} + \frac{(\nabla\phi)^2}{2} + \frac{m^2\phi^2}{2}$ and $\rho_\chi = e^{\gamma\phi}\frac{a^2\dot{\chi}^2+(\nabla\chi)^2}{2a^2}$.  In the top row we have $\gamma M_P = 5$ and in the bottom row we have $\gamma M_P = 6.25$.  The color coding of the power spectrum lines are the same as~\figref{fig:exponent_spectra_phi4}.}
\end{figure}


\subsection{$G(\phi)=e^{\gamma\phi^2}$}
\begin{equation}
  G(\phi) = e^{\gamma\phi^2}
  \label{eqn:coupling_gaussian}
\end{equation}
\begin{equation}
  \omega_k^2 = k^2 - \gamma\ddot{\phi}_{bg}\phi_{bg} - (\gamma^2\phi_{bg}^2+\gamma)\dot{\phi}_{bg}^2
\end{equation}
\begin{figure}
  \includegraphics[width=0.32\linewidth]{{{floquet_sine_gaussian_mass0}}}
  \includegraphics[width=0.32\linewidth]{{{floquet_sine_gaussian_mass0.01}}}
  \includegraphics[width=0.32\linewidth]{{{floquet_sine_gaussian_mass0.1}}}
  \caption{Floquet charts for the gaussian noncanonical coupling $G(\phi)=e^{\gamma\phi^2}$ for several choices of $m_\chi^2$ assuming background oscillations $\phi_{bg}=\phi_0\cos(m_\phi t)$.}
\end{figure}

\begin{figure}
  \includegraphics[width=0.45\linewidth]{{{energy_grad_vary_alpha_gaussian_m2_n128_l20}}}
  \caption{Fraction of energy stored in $\chi$ as defined in~\eqref{eqn:rho_chi_phi} (\emph{solid line}) and gradient energy (\emph{dashed line} as we vary the parameter $\alpha M_P^2$ for the gaussian noncanonical coupling $G(\phi)=e^{\alpha\phi^2}$.  The $\phi$ potential was $V(\phi)=m_\phi^2 \phi^2 / 2$ and we used $N_{lat}=128^3$ with a comoving side length of $m_\phi L = 20$.}
\end{figure}

\subsection{$G(\phi)=(1+\xi\phi^2)^{-1}$}
\begin{equation}
  G(\phi) = \frac{1}{1+\xi\phi^2}
\end{equation}
\begin{equation}
  \omega_k^2 = k^2 + \frac{\xi\phi_{bg}}{1+\xi\phi_{bg}^2}\ddot{\phi}_{bg} - \left(\frac{2\xi^2\phi_{bg}^2-\xi}{(1+\xi\phi_{bg}^2)^2} \right)\dot{\phi}_{bg}^2
  \label{eqn:frequency_conformal_flatten}
\end{equation}
\begin{figure}
  \includegraphics[width=0.45\linewidth]{{{floquet_sine_flatten_mass0_big}}}
  \includegraphics[width=0.45\linewidth]{{{floquet_sine_flatten_mass0.01_big}}} \\
  \includegraphics[width=0.45\linewidth]{{{floquet_sine_flatten_mass0.1_big}}}
  \includegraphics[width=0.45\linewidth]{{{floquet_sine_flatten_mass1_big}}}
  \caption{Floquet chart for the oscillations with effective frequency given by~\eqref{eqn:frequency_conformal_flatten}.  Once again, the left panel corresponds to a background oscillation in a quadratic potential and the right panel to background oscillations in a quartic potential.}
\end{figure}
\begin{figure}
  \includegraphics[width=0.45\linewidth]{{{energy_grad_vary_xi_flatten_m2_n128_l10}}}
  \includegraphics[width=0.45\linewidth]{{{energy_grad_vary_m_flatten_m2_n128_l10}}}
  \caption{Fraction of energy stored as $\rho_\chi$ and $\rho_{grad}$ for background inflaton oscillations in an $m_\phi^2\phi^2/2$ potential with the noncanonical coupling~\eqref{eqn:frequency_conformal_flatten}.  In the left figure we vary $\xi M_P^2$ while holding $m_\chi^2 = 0$ and in the right figure we vary $m_\chi^2/m_\phi^2$ while holding $\xi M_P^2 = 5000$ fixed.}
\end{figure}

\begin{figure}
  \includegraphics[width=0.32\linewidth]{{{weff_flatten_xi1000_l10}}}
  \includegraphics[width=0.32\linewidth]{{{phi_spec_flatten_xi1000_l10}}}
  \includegraphics[width=0.32\linewidth]{{{chi_spec_flatten_xi1000_l10}}}
  \caption{Evolution of the equation of state and fluctuations in $\phi$ and $\chi$ for coupling $G(\phi)=(1+\xi\phi^2)^{-1}$ with $\xi M_P^2=1000$ and $V(\phi)=m^2\phi^2/2$.  The color coding of the power spectrum lines are the same as~\figref{fig:}.}
\end{figure}

\subsection{$G(\phi) = 1+\xi\phi^2$}
\begin{equation}
  G(\phi) = 1 + \xi\phi^2
\end{equation}
\begin{equation}
  \omega_k^2 = k^2 - \frac{\xi\phi_{bg}}{1+\xi\phi_{bg}^2}\ddot{\phi}_{bg} - \left( \frac{\xi}{(1+\xi\phi_{bg}^2)^2} \right)\dot{\phi}_{bg}^2
  \label{eqn:frequency_quadratic_kinetic}
\end{equation}
\begin{figure}
  \includegraphics[width=0.32\linewidth]{{{floquet_sine_running_mass0}}}
  \includegraphics[width=0.32\linewidth]{{{floquet_sine_running_mass1}}}
  \includegraphics[width=0.32\linewidth]{{{floquet_sine_running_mass10}}}
  \caption{Floquet chart for~\eqref{eqn:frequency_quadratic_kinetic} with $\phi_{bg} = \phi_0\cos(m_\phi t)$ for several choices of the preheat field mass $m_\chi^2$.}
\end{figure}
\begin{figure}
  \includegraphics[width=0.45\linewidth]{{{energy_grad_vary_xi_running_m2_n128_l20}}}
  \includegraphics[width=0.45\linewidth]{{{energy_grad_vary_m_running_m2_n128_l20}}}
  \caption{Partition of energy within our simulation volume as we vary the noncanonical coupling $\xi$ with $m_\chi^2 = 0$ (\emph{left}) and as we vary the mass $m_\chi^2/m_\phi^2$ for fixed noncanonical coupling $\xi M_P^2 = 5000$ (\emph{right}).  Both plots used an inflaton potential $V^{inf}(\phi) = m_\phi^2\phi^2/2$.  The solid lines are the fraction of the total energy in the $\chi$ field, while the dashed semitransparent lines are the fraction of the total energy stored as gradient energy.  All simulations used lattices with $N_{lat}=128^3$ and comoving side length $m_\phi L = 20$.}
\end{figure}


\section{Direct Potential Couplings and Caustic Formation}
{\bf How much of this should go in here and how much in the ballistics paper?}
Now that we have studied the effects of noncanonical couplings between the oscillating inflaton and preheat fields, 
we return to the more familiar case of a direct potential couplings between the inflaton and the preheating fields.
For this section we assume that the fields have canonical kinetic terms and that we can expand the potential around the minimum in a low order polynomial
\begin{equation}
  V(\phi,\chi_i) = 
\end{equation}
As a special case due to the very intriguing expansion history phenomenology,
we will also consider minima for which there are only dimensionless couplings.
We are especially interested in cases where the bottom of the potential posseses some flat (or nearly flat) directions.
We consider two distinct cases:
\begin{itemize}
  \item Many separate fields all couple to the inflaton
  \item A single field couples to the inflaton, which then couples to other fields
\end{itemize}

\subsection{Direct Couplings to Many Fields}
For the first case, we thus have potentials of the form
\begin{equation}
  V(\phi,\chi_i) = \frac{\lambda_\phi}{4}\phi^4 + \sum\frac{g_i^2}{2}\phi^2\chi_i^2 + \sum_i\frac{\lambda_i}{4}\chi_i^4 
\end{equation}
where the $g_i^2$ have some distribution.
In general, there should be additional nondiagonal cross-couplings between the $\chi$ fields, but we will not consider these.
For the case of a single $\chi_i$, there are special values of $g^2/\lambda$ such that the zero mode of the $\chi$ field is unstable.
If a mechanism exists to produce large-scale isocurvature fluctuations in $\chi$ prior to the start of preheating, then these large scale perturbations can be converted into adiabatic density perturbations.
The simple question we will consider is the following, suppose that we have many $\chi_i$'s with couplings $g_i^2$ drawn from some distribution.
We now demonstrate that the same spiky pattern of density perturbations can be produced provided we have a single $\chi_i$ whose zero-mode is unstable.
{\bf Question: what if the arm for that $\chi_i$ is cut off.  Can the particle still bounce up into another arm}
{\bf I've got homogeneous runs showing the caustic formation done so far, but lattice runs will presumably confirm this}
{\bf More generally we would like to have precise conditions that allow the production of adiabatic density perturbations, but this seems rather hard in the general case}
{\bf If we take a large number of $\chi_i$'s there is an enhancement of the rate of production of fluctuations.  Does this lead to an earlier shock, and if it does do we also lose the spikes?}

\subsection{Direct Coupling to a Single Field, Coupled to Many Fields}
For the second case we consider field potentials of the form
\begin{equation}
  V(\phi,\sigma_i)  \frac{\lambda}{4}\phi^4 + \frac{g^2}{2}\phi^2\left(\sum \alpha_i\sigma_i \right)^2 + V(\sigma_i)
\end{equation}
with $\sum_i \alpha_i^2 = 1$ and $g^2\lambda$ chosen such that the zero mode is resonant.
Thus, the inflaton couples to only a single effective field $\tilde{\sigma} \equiv \sum_i\alpha_i\sigma_i$.
We will further engineer $V(\sigma_i)$ to posess some arm like structures (ie. flat directions) that aren't necessarily aligned with the field $\tilde{\sigma}$.
{\bf What are the conditions on V to get spikes?  How to expand V?}

As a special case, we can consider the three field model
\begin{equation}
  V(\phi,\sigma_1,\sigma_2) = \frac{\lambda}{4}\phi^4 
  + \frac{g_1^2}{2}\phi^2(\cos\theta\sigma_1+\sin\theta\sigma_2)^2
  + \frac{g_\sigma^2}{2}\sigma_1^2\sigma_2^2 
  + \frac{\lambda_1}{4}\sigma_1^4
  + \frac{\lambda_2}{4}\sigma_2^4
  \label{eqn:3field_flatdir}
\end{equation}
with $\theta$,$g_i^2$,$\lambda_i$ model parameters.
When $\lambda_1=0=\lambda_2$, the $\phi$ independent part of the potential has a cross structure with two long arms along the $\sigma_1$ and $\sigma_2$ directions.
If the fields make large excursions down these arms then spiky density perturbations can be produced.
For $\lambda_i \neq 0$, these arms are cutoff and the homogeneous field dynamics is trapped in the bottom of the potential.
By adjusting $\theta$, we change the alignment between the unstable field direction and the flat directions.

\begin{figure}
  \includegraphics[width=0.32\linewidth]{{{caustics_cross_l1_lcut0}}}
  \includegraphics[width=0.32\linewidth]{{{caustics_cross_l1_lcut0.5}}}
  \includegraphics[width=0.32\linewidth]{{{caustics_cross_l1_lcut1}}}
  \caption{Evolution of homogeneous trajectories in the potential~\eqref{eqn:3field_flatdir}.  In all cases we took $\theta = \pi/4$, $g_1^2=2\lambda$ and $g_\sigma^2 = 2\lambda$.  From left to right we have $\lambda_2 = 0, 0.5, 1$.  The horiontal axis gives the initial amplitude of $\sigma_1^{init}$ and we took $\sigma_2^{init}=\sigma_1^{init}/10$. The color scale is $a^2H$, with white corresponding to the late-time mean over all trajectories.  {\bf What happens if we change initial conditions in  $\sigma_1,\sigma_2$ plane}}
\end{figure}

\section{Specific Models?}
Do running kinetic Higgs model and nonminimally coupled gravity with conformal flattenning here?
\subsection{Running Kinetic Higgs Inflation}
\begin{equation}
  \mathcal{L} = -\frac{1}{2}\left(1+\kappa\phi^2\right)\partial_\mu\phi\partial^\mu\phi - \lambda\frac{\phi^4}{4}
\end{equation}

\section{Nonminimal Gravitational Coupling}
A general class of models which can lead to the noncanonical couplings described above are scalar fields nonminimally coupled to gravity with Jordan frame action
\begin{equation}
  \frac{\mathcal{L}_J}{\sqrt{|\hat{g}|}} = \frac{M_P^2}{2}f^2(\phi)\hat{R} - \frac{\partial_\mu\phi\partial^\mu\phi}{2} - \frac{\partial_\mu\chi\partial^\mu\chi}{2} - \lambda V^{inf}(\phi) - V(\phi,\chi) \, .
\end{equation}
We restrict ourselves to cases where only a single field has a nonminimal gravitational coupling and for this paper we take that field to be the inflaton.
After performing a conformal transformation $g_{\mu\nu} = f^2(\phi)\hat{g}_{\mu\nu}$, we obtain the Einstein frame action
\begin{equation}
  \frac{\mathcal{L}_E}{\sqrt{|g|}} = \frac{M_P^2}{2}R -\frac{1+6(\partial_\phi f)^2}{f(\phi)^2} \frac{\partial_\mu\phi\partial^\mu\phi}{2} - \frac{\lambda V^{inf}(\phi)}{f(\phi)^4} - \frac{1}{2f(\phi)^2}\partial_\mu\chi\partial^\mu\chi - \frac{V(\phi,\chi)}{f(\phi)^4}
\end{equation}
A subclass of these models that have received considerable attention recently impose the additional constraint that $f^2(\phi) = 1 + \xi\sqrt{V^{inf}(\phi)}$,
which yields universal predictions for the cosmological parameters $n_s=1-8/N$ {\bf Check this} and $r=12/N_e^2$ in the $\xi \gg 1$ limit.

\subsection{Higgs Inflation}
{\bf Possible thing to look at, can we make oscillons in this model since the potential flattens out away from the minimum?}
One well studied model of precisely this form is Higgs inflation with $f(\phi)^2 = 1+\xi\phi^2$.
From the preheating point of view, this model is particularly interesting as all of the interactions between the Higgs and other SM degrees of freedom are known.
Therefore, under the assumption that no additional degrees of freedom are excited during the stage of oscillations about the minimum, this provides a predictive model for the preheating dynamics.
{\bf Of course, this involves studying gauge-fields and fermions, not production of scalar fields.  The gauge fields should qualitatively behave the same as scalars, but fermions will have Fermi blocking effects at large occupation numbers (perhaps leading to the formation of a condensate from preheating?).  We won't tackle these issues here.}

The initial field values were determined by the potential slow-roll conditions $\epsilon_V=1$
\begin{equation}
  \phi_{end}^2 = \frac{\sqrt{1+32\xi(1+6\xi)}-1}{2\xi(1+6\xi)} \qquad \dot{\phi}_{end} = -\frac{2\phi_{end}}{\sqrt{3}(1+\xi(1+6\xi)\phi_{end}^2)} \, .
\end{equation}
For $\xi \gtrsim 1$, we have $\lambda^{-1/2} \approx 47000/\xi$ to obtain the correct amplitude of primordial fluctuations during inflation, with the overall constant depending somewhat on our assumptions about the post-inflation evolution (specifically the number of efolds).
{\bf This should be a temporary approach.  Better choice is to numerically evolve to determine initial conditions along attractor at $\epsilon = 1$ then start the simulations from there (or even do the linear part of the resonance separately, although then the fields and momenta become correlated initially).}
In~\figref{fig:energy_partition_higgs} we plot the fraction of the energy stored in gradients as we vary $\xi M_P^2$ while simultaneously tuning $\lambda$ to maintain the same amplitude of primordial scalar fluctuations.  Without direct potential couplings to the field $\chi$, only $\phi$ develops large inhomogeneities.
This is in accord with our previous results, where we found that we required $\xi \phi_0^2 \sim 10^3$ in order to excite the $\chi$ field with the noncanonical coupling appropriate for the conformally flattenned Higgs model, albeit with a different background $\phi$ evolution.
{\bf Make sure that $\chi$ remaining unexcited isn't an artifact of the chosen size of the simulation cube.}

\begin{figure}
  \includegraphics[width=0.45\linewidth]{{{conformal_flatten_potentials_n60}}}
  \caption{Potential seen by the canonical field $\phi_{canonical}$ driving inflation with $\lambda$ adjusted to give the correct amplitude for the scalar power spectrum, assuming the appropriate scale exited $N_e=60$ efolds before the end of inflation.}
\end{figure}

\begin{figure}
  \caption{Evolution of the background oscillations in $\phi$ for $\xi=10$ and $\xi=100$.  The stage of oscillations in a $\phi^4$ leading to a damped Jacobi elliptic cosine is preceded by a stage where the field feels the nonquartic nature of the potential away from the minimum.}
\end{figure}

\begin{figure}
  \caption{Evolution of effective mass for the $\phi$ and $\chi$ fluctuations for several choices of $\xi$ and couplings to $\chi$.}
\end{figure}

\begin{figure}
  \includegraphics[width=0.5\linewidth]{{{energy_grad_vary_xi_higgs_n32}}}
  \caption{Evolution of energy densities for several choices of $\xi$ coupled to a massless $\chi$ with no direct potential couplings.  The solid lines are the gradient energy in $\phi$ and the dashed lines the gradient energy in $\chi$.  In all cases considered, only the $\phi$ field experiences parametric resonance and becomes inhomogeneous.}
  \label{fig:energy_partition_higgs}
\end{figure}

\begin{figure}
  \caption{Evolution of field spectra for several choices of $\xi$ and couplings to $\chi$.}
\end{figure}
For $\xi \ll 1$ this model reduces to $\lambda\phi^4$.  For $\xi \gg 1$, the effective coupling constants arising from the noncanonical kinetic term are $\xi\phi_{end}^2 \sim 1$, so that strong resonance of $\chi$ does not occur solely from the minimal couplings arising from the conformal transformation.
However, as with the $\xi=0$ case, a coupling $g^2\phi^2\chi^2$ (or $\sigma\phi\chi^2$) will drive the growth of fluctuations in $\chi$.

%{\bf Further, for large $\xi$, we have a much larger $\lambda$.  This further suppresses the effective couplings, which are $\phi_0/M_{osc}$.} 
%Some things to keep in mind, for $\xi \gg 1$, we have $\lambda \propto \xi^2$.
%The initial fluctuation amplitudes scale as $\sqrt{\lambda}$ while the initial oscillation amplitude for $\phi$ scales as $\xi^{-1}$ {\bf check this one}.
%The strength of the instability scales as {\bf check this} so that we expect the fluctuations to become nonlinear at {\bf is this modified for the new coupling?}.

\section{Dynamics in a Decoupled Sector}
Take $\mathcal{L}_{M}$ to have no direct couplings to the inflaton (except those induced by the conformal transformation into the Einstein frame).
What are the possible dynamics?  Fields become light during inflation for some forms of $f(\phi)^2$, so they can develop large fluctuations.  What happens when the interactions turn on due to evolution of $\phi$?

\section{Appendix}
Brief description of initial conditions and lattice discretization.
Due to the noncanonical nature of the fields, our prescription for initialization of fluctuations is slightly modified from the canonical case.
\subsection{Initial Conditions}
Denoting a generic field by $\phi$ we have
\begin{equation}
  \phi_{nc} = \phi_{nc}(\phi_{c}) \approx \bar{\phi_{nc}} + \frac{\partial\phi_{nc}}{\partial\phi_{c}}\delta\phi_{c} \implies \delta\phi_{nc} \approx \frac{\partial\phi_{nc}}{\partial\phi_{c}}\delta\phi_{c}
\end{equation}
For the $\chi$ field (where it's noncanonical piece is determined by another field), we set $\chi_c = \sqrt{G(\phi)}\chi$ as the canonically normalized field (and since $a=1$ at the start of our simulations we can ignore it in the normalization).
{\bf This isn't quite right since this transformation will induce an additional terms into the kinetic piece for $\phi$, which then modifies the $\phi$ initial conditions (which I'm not accounting for)}
As with the standard approach, we take $\delta\phi_{c}$ and $\delta\dot{\phi}_{c}$ to be independent Gaussian random fields with spectra $\frac{1}{\sqrt{2\omega_k}}$ and $\delta\dot{\phi} = \sqrt{\omega_k/2}$ respectively, where $\omega_k$ is the effective frequency of the canonically normalized field.\footnote{Treating the field and its momentum as uncorrelated is correct on subhorzon scales, but for superhorizon scales the squeezing of the modes into a two-mode squeezed state induces large correlations between $\phi$ and $\Pi$.  In a proper treatment these must be accounted for.}

\subsection{Lattice Discretization}
{\bf Use better notation, and check factors of $M_P$.}
We evolve Hamilton's equations for our latticized field system.
The equations of motion are obtained by directly discretizing the action then performing a Legendre transform to obtain the corresponding Hamiltonian.

For the metric $ds^2=-dt^2 + a^2(t)dx^2$, %(with Ricci scalar $R=-6\frac{\ddot{a}}{a}+H^2}$)
we take the metric variables to be $y_a = a^{3/2}$ so that $\Pi_a = -8M_P^2N_{lat}\dot{y}_a/3=-4N_{lat}Ha^{3/2}$, leading to the Hamiltonian
\begin{equation}
  \mathbb{H} = \int dt \sum_i \left[y_a^2\left(\Pi^A_i\frac{G^{-1}_{AB}}{2y_a^4}\Pi^B_i + \nabla\phi^A_i\frac{G_{AB}}{2y_a^{4/3}}\nabla\phi_i^B + V\right) - \frac{3\Pi_a^2}{16M_P^2N_{lat}^2} \right]
\end{equation}
From this we derive Hamilton's equations
\begin{align}
  \frac{d \phi^A_i}{dt} &= G^{-1}_{AB}\frac{\Pi^B}{y_a^2} \\
  \frac{d \Pi^A_i}{dt} &= -\frac{1}{2y_a^2}\Pi^T\frac{\delta G^{-1}}{\delta\phi^A}\Pi - y_a^2\partial_{\phi^A}V  + y_a^{2/3}G\nabla^2\phi^A - \frac{y_a^{2/3}}{2}\nabla\phi^T\frac{\delta G}{\delta \phi^A}\nabla\phi \\
  \frac{d y_a}{dt} &= -\frac{3\Pi_a}{8N_{lat}}  \\
  N_{lat}^{-1}\frac{d \Pi_a}{dt} &= \frac{1}{N_{lat}}\left(\sum_i \frac{\Pi^TG^{-1}\Pi^t}{y_a^3} - 2y_aV - y_a\frac{1}{3a^2}\nabla\phi^TG\nabla\phi \right) = 2a^{3/2} \langle P \rangle
\end{align}
With the metric $ds^2 = a^2(\tau)(-d\tau^2 + dx^2)$ %(with Ricci scalar $R=6\frac{a''}{a^3}$)
we instead choose $y_a = a$ so $\Pi_a = -6N_{lat}a^2H = -6N_{lat}y_a'$
with Hamiltonian
\begin{equation}
  \mathbb{H} = y_a^4\int dt \sum_i \left[ \frac{1}{2y_a^6}\Pi_i^TG^{-1}\Pi_i + \frac{1}{2y_a^2}(\nabla\phi)^TG(\nabla\phi) + V - \frac{\Pi_a^2}{12M_P^2N_{lat}^2}\right]
\end{equation}
The corresponding equations of motion are
\begin{align}
  \frac{d \phi^A_i}{d\tau} &= G^{-1}_{AB}\frac{\Pi^B}{y_a^2} \\
  \frac{d \Pi^A_i}{d\tau} &= -\frac{1}{2y_a^2}\Pi^B_i\frac{\delta G^{-1}_{BC}}{\delta \phi_i^A}\Pi_i^C - y_a^4\partial_{\phi^A}V - \frac{y_a^2}{2}(\nabla\phi)^T\frac{\delta G}{\delta \phi^A}(\nabla\phi) + y_a^2G\nabla^2\phi\\
  \frac{d y_a}{d\tau} &= -\frac{\Pi_a}{6N_{lat}} \\
  N^{-1}_{lat}\frac{d \Pi_a}{d\tau} &= \frac{1}{N_{lat}}\sum_i \frac{\Pi^TG^{-1}\Pi}{y_a^3} - 4y_a^3V - y_a\nabla\phi^TG\nabla\phi
\end{align}

The only subtlety in the above procedure comes from our discretization of  $\nabla\phi^TG(\phi)\nabla\phi$.
Our prescription for dealing with the derivatives of the fields is to discretize the gradient energy as
\begin{equation}
  \nabla\phi_IG^{IJ}(\vec{\phi})\nabla\phi_J \equiv \sum_{\alpha} d_{\alpha}G(\phi(x_i))(\phi_I(x_{i+\alpha})-\phi_I(x_i))(\phi_J(x_{i+\alpha})-\phi_J(x_i))
\end{equation}
where the coefficients $d_\alpha$ represent a choice of spatial discretization of the system, and we assume a symmetric stencil $d_{-\alpha}=d_\alpha$.
By working directly with the discretized Hamiltonian, we see that self-consistency requires the laplacian operator be defined as
%\begin{equation}
%\nabla\phi^T\frac{\delta G}{\partial \delta\phi_i}\nabla\phi =
%\end{equation}
%and
\begin{equation}
  G_{BA}(\phi)\nabla^2\phi^A \to \sum_{\alpha}2d_\alpha \frac{G_{BA}(\phi_{i+\alpha})+G_{BA}(\phi_i)}{2}\left(\phi^A_{i+\alpha}-\phi^A_i\right) \, .
\end{equation}

\end{document}
